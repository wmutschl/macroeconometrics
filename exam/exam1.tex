% !TeX encoding = UTF-8
% !TeX spellcheck = en_US
\documentclass{article}
%%%%%%%%%%%%%%%%%%%%%%%%%%%%%%%%%%%%%%%%%%%%%%%%%%%%%%%%%%%%%%%%%%%%%%%%%%%%%%%%%%%%%%%%%%%%%%%%%%%%%%%%%%%%%%%%%%%%%%%%%%%%%%%%%%%%%%%%%%%%%%%%%%%%%%%%%%%%%%%%%%%%%%%%%%%%%%%%%%%%%%%%%%%%%%%%%%%%%%%%%%%%%%%%%%%%%%%%%%%%%%%%%%%%%%%%%%%%%%%%%%%%%%%%%%%%
\usepackage[a4paper,top=2cm]{geometry}
\usepackage{amssymb,amsmath,amsfonts}
\usepackage[english]{babel}
\usepackage[a4paper]{geometry}
\usepackage{enumitem}
\usepackage{booktabs}
\usepackage{csquotes}
%\parindent0mm
%\parskip1.5ex plus0.5ex minus0.5ex

\begin{document}

\title{Macroeconometrics}
\author{Exam No. 1 of 3}
\date{Winter 2017/2018}
\maketitle

\begin{itemize}
\item Answer \textbf{all} of the following four exercises in either German or English.

\item Hand in your solutions before Tuesday November, 7 2017 at 14:00.

\item Please e-mail the solutions file to willi@mutschler.eu. 

\item The solution files should contain your executable (and commented) Matlab functions and script files as well as an additional documentation preferably as \texttt{pdf}, not \texttt{doc} or \texttt{docx}. 

\item I will confirm the receipt of your work also by email.

\item All students must work on their own, please also give your student ID number.

\item It is advised to regularly check the learnweb in case of urgent updates.

\item If there are any questions, do not hesitate to contact Willi Mutschler.
\end{itemize}
\newpage

\section{AR(1) with time trend}
Consider the AR(1) model with constant and time trend
$$ y_t = c + d\cdot t + \phi y_{t-1} + u_t$$
where $u_t$ is $iid(0,\sigma^2)$, $|\phi|<1$, $c \in \mathbb{R}$ and $d \in \mathbb{R}$.
\begin{enumerate}
	\item Compute the unconditional first and second moments, i.e. the unconditional mean, variance, autocovariance and autocorrelation of $y_t$.
	\item Why is this process not covariance-stationary? How could one proceed to make it covariance-stationary?
\end{enumerate}
\newpage
\section{Portmanteau Test For Residual Autocorrelation}
The portmanteau test checks the null hypothesis that there is no remaining residual autocorrelation at lags $1$ to $h$ against the alternative that at least one of the autocorrelations is nonzero. In other words, the pair of hypotheses:
\begin{align*}
H_0:\rho_u(1)=\rho_u(2)=...=\rho_u(h) = 0
\end{align*}
versus:
\begin{align*}
H_1:\rho_u(j)\neq 0 \text{ for at least one }j=1,...,h
\end{align*}
is tested, where $\rho_u(j) = Corr(u_t, u_{t-j})$ denotes an autocorrelation coefficient of
the residual series. Consider the Box-Pierce test statistic $Q_h$
\begin{align*}
Q_h = T \sum_{j=1}^h \hat{\rho}^2_u(j)
\end{align*}
which has an approximate $\chi^2(h-p)$-distribution if the null hypothesis holds and $T$ is the length of the residual series. The null hypothesis of no residual autocorrelation is rejected for large values of the test statistic. 

\begin{enumerate}
	\item Load Quarterly data for the price index of US Gross National Product given in gnpdeflator.txt by simply calling \texttt{load gnpdeflator.txt}. This is a chain-type price index with basis year 2005. The data is seasonally adjusted and spans from 1954.Q4 to 2007.Q4.
	\item Compute the inflation series. That is, take the first difference of log(gnpdeflator).
	\item Use the Akaike information criteria to determine the lag length $\hat{p}$.
	\item Estimate two models: (i) an $AR(\hat{p})$ model and (ii) an $AR(1)$ model with OLS.
	\item Set $h=\hat{p}+10$ and compute $Q_h$ as well as the corresponding p-value (\texttt{chi2pdf($Q_h$,h-p)}) for both models.
	\item Comment, based on your findings, whether the residuals are white noise.
\end{enumerate}

%\begin{solution}
%The size of the test may be
%unreliable if h is too small, and it may have reduced power if h is large and,
%hence, many \enquote{noninformative} autocorrelations are included. Also, it has been
%found in Monte Carlo studies that the $\chi^2$ approximation to the null distribution of the test statistic is a good one only for very large sample sizes T . Therefore,
%Ljung and Box (1978) have proposed a modified version of the portmanteau
%statistic for which the approximation was found to be more suitable in some
%situations. Clearly, remaining residual autocorrelation indicates a model defect. It may be
%worth trying a model with larger orders in that case.
%
%\end{solution}
\newpage
\section{Maximum Likelihood Estimation}
Consider the AR(1) model with constant
$$ y_t = c + \phi y_{t-1} + u_t$$
Assume that the error terms $u_t$ are i.i.d. Laplace distributed with known density
\begin{align*}
f_{u_{t}}(u)=\frac{1}{2}\exp \left( -|u|\right)
\end{align*}
Note that $E(u_t)=0$ and $Var(u_t)=2$.
\begin{enumerate}
		\item Derive the log-likelihood function conditional on the first observation.
		\item Write a MATLAB function that calculates the conditional log-likelihood of $c$ and $\phi$.
		\item Load the dataset \texttt{LaPlace.txt} by running \texttt{load LaPlace.txt}.
		\item Numerically find the maximum likelihood estimates of $c$ and $\phi$ by minimizing the negative conditional log-likelihood function.
		\item Compare your results with the maximum likelihood estimate under the assumption of Gaussianity. That is, redo the estimation by minimizing the negative Gaussian log-likelihood function.
	\end{enumerate}
\newpage


\section{Bootstrap Test Statistics}

Consider the AR(1) model with constant
\begin{equation*}
y_{t}=c +\phi y_{t-1}+u_{t}
\end{equation*}
for $t=1,\ldots ,T$ with i.i.d. error terms $u_{t}$ and $E(u_{t}|y_{t-1})=0$.
Usually, we construct a $95\%$-confidence interval for e.g. $\phi$ using the normal approximation
\begin{equation*}
\left[ \hat{\phi}-1.96\cdot SE(\hat{\phi});\ \hat{\phi}+1.96\cdot SE(\hat{\phi})\right]
\end{equation*}
with $\hat{\phi}$ denoting the OLS estimate and $SE(\hat{\phi})$ the estimated standard error of $\phi$. If one does not know the asymptotic distribution of a test statistic (or it has a very complicated form), one often relies on a nonparametric approach. To this end, we are going to \enquote{bootstrap}, i.e. recompute the t-statistics a large number of times on artificial data generated from resampled residuals.\\
We will do this step-by-step, i.e. write a Matlab script for the following:
\begin{itemize}
	\item Simulate $T=100$ observations with $c=1$, $\phi=0.8$ and errors drawn from the exponential distribution using $\texttt{u = exprnd(1,T,1)-1}$ such that $E(u_t)=0$.
	\item Estimate the model with OLS and calculate the t-statistic $\tau=\frac{\hat{\phi}}{SE(\hat{\phi})}$. 
	\item Store the OLS residuals in a vector $\hat{u} = (\hat{u}_{2},\ldots ,\hat{u}_{T})'$.
	\item Set $B=10000$ and initialize the output vector $\tau^{\ast} = (\tau_1^\ast,...,\tau_B^\ast)$. \item For $b=1,...,B$:
	\begin{itemize}
	\item Draw a sample \textbf{with replacement} from $\hat{u}$ and save it as $u^{\ast} = u_{2}^{\ast},\ldots ,u_{T}^{\ast }$.\\Hint: For sampling with replacement use either the \texttt{datasample} or \texttt{randi} function. If you don't have the necessary toolbox installed you can also use \\\texttt{ustar = uhat(ceil(size(uhat,1)*rand(T-1,1)),:)}.
	\item Initialize an artificial time series $y_t^\ast$ with $T$ observations and set $y_1^\ast = y_1$.
	\item For $t=2,\ldots ,T$ generate
	\begin{equation*}
	y_{t}^{\ast }=\hat{c}+\hat{\phi}y^\ast_{t-1}+u_{t}^{\ast }
	\end{equation*}
	\item Estimate an AR(1) model on this artificial dataset with OLS. Store the following t-statistic in your output vector at position b:
	$$\tau^\star = \frac{\phi^\star - \hat{\phi}}{SE(\phi^\ast)}$$
	\end{itemize}
	\item Sort the output vector using \texttt{sort} such that $\tau_{(1)}^\ast \leq ... \leq \tau_{(B)}^\ast$.
	 \item  The bootstrap approximate confidence interval for $\phi$ is then
	\begin{equation*}
	\left[ \hat{\phi}-\tau_{((1-\alpha /2)R)}^{\ast }\cdot SE(\hat{\phi});\ \hat{\phi}-\tau_{((\alpha/2)R)}^{\ast }\cdot SE(\hat{\phi})\right] 
	\end{equation*}
	Set $\alpha=0.05$ and compare this with the normal approximation.
	\item Redo the exercise for $T=30$ and $T=10000$. Comment on your findings.
\end{itemize}

%Typischerweise führen asymptotische
%Resultate, sofern verfügbar, für entsprechende Schätz- und Testverfahren zu Grenzverteilungen,
%die nichtstandard sind und/oder von unbekannten Parametern abhängen, die nur schwer zu
%schätzen sind. Darüber hinaus zeigen asymptotische Verfahren bei kleineren Stichproben häufig
%schlechte Approximationseigenschaften. Bootstrapverfahren helfen in beiden Situationen, die
%Verteilungen statistischer Größen zu schätzen und bei kleinen Stichproben dabei bessere Er-
%gebnisse zu erzielen. Gleichzeitig hat sich die Praktikabilität dieser Verfahren durch den Einsatz
%von Computern mit immer stärkerer Rechenleistung deutlich verbessert.
%Besonders aufgrund der flexiblen und vielfältigen Einsatzmöglichkeiten der Bootstrap-Methodik,
%sehe ich großes strategisches Forschungspotential an der Fakultät für Statistik.

\end{document}
