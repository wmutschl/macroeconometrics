% !TeX encoding = UTF-8
% !TeX spellcheck = en_US
\documentclass{article}
%%%%%%%%%%%%%%%%%%%%%%%%%%%%%%%%%%%%%%%%%%%%%%%%%%%%%%%%%%%%%%%%%%%%%%%%%%%%%%%%%%%%%%%%%%%%%%%%%%%%%%%%%%%%%%%%%%%%%%%%%%%%%%%%%%%%%%%%%%%%%%%%%%%%%%%%%%%%%%%%%%%%%%%%%%%%%%%%%%%%%%%%%%%%%%%%%%%%%%%%%%%%%%%%%%%%%%%%%%%%%%%%%%%%%%%%%%%%%%%%%%%%%%%%%%%%
\usepackage[a4paper,top=2cm]{geometry}
\usepackage{amssymb,amsmath,amsfonts}
\usepackage[english]{babel}
\usepackage[a4paper]{geometry}
\usepackage{enumitem}
\usepackage{booktabs}
\usepackage{csquotes}
%\parindent0mm
%\parskip1.5ex plus0.5ex minus0.5ex

\begin{document}
	
	\title{Macroeconometrics}
	\author{Exam No. 2 of 3}
	\date{Winter 2017/2018}
	\maketitle
	
	\begin{itemize}
		\item Answer \textbf{all} of the following four exercises in either German or English.
		
		\item Hand in your solutions before Monday January, 8 2018 at 12:00.
		
		\item Please e-mail the solutions files to willi@mutschler.eu. 
		
		\item The solution files should contain your executable (and commented) Matlab functions and script files as well as an additional documentation preferably as \texttt{pdf}, not \texttt{doc} or \texttt{docx}. 
		
		\item I will confirm the receipt of your work also by email.
		
		\item All students must work on their own, please also give your student ID number.
		
		\item It is advised to regularly check the learnweb and your emails in case of urgent updates.
		
		\item If there are any questions, do not hesitate to contact Willi Mutschler.
	\end{itemize}
	\newpage
	
\section{Properties Of Lag-Order Selection Criteria}
Assume that the true Data-Generating-Process (DGP) follows the following VAR(4) model
$$y_t = \begin{pmatrix}
2.4 & 1.0\\
0 & 1.1
\end{pmatrix}
y_{t-1}+
\begin{pmatrix}
-2.15 & -0.9\\
0 & -0.41
\end{pmatrix}
y_{t-2}+
\begin{pmatrix}
0.852 & 0.2\\
0& 0.06
\end{pmatrix}
y_{t-3}+
\begin{pmatrix}
-0.126 & 0\\
0 & 0.0003
\end{pmatrix}
y_{t-4}
+ u_t$$
where $u_t$ is a Gaussian white noise with contemporary covariance matrix $\Sigma_u = \begin{pmatrix}
0.9&0.2\\0.2&0.5
\end{pmatrix}$.

Perform a Monte-Carlo analysis to study both the finite-sample as well as asymptotic properties of the Akaike Information Criterion (AIC) and the Schwarz Information Criterion (SIC):
\begin{align*}
AIC(m)  &= \log(det(\tilde{\Sigma}_u(m))) + \frac{2}{T}\varphi(m)\\
SIC(m)  &= \log(det(\tilde{\Sigma}_u(m))) + \frac{\log T}{T}\varphi(m)
\end{align*}
where $\tilde{\Sigma}_u=T^{-1}\sum_{t=1}^T \hat{u}_t\hat{u}_t'$ is the residual covariance matrix estimator for a reduced-form VAR model of order $m$ based on OLS residuals $\hat{u}_t$. The function $\varphi(m)$ corresponds to the total number of regressors in the system of VAR equations. The VAR order is chosen such that the respective criterion is minimized over the possible orders $m = 0,...,p^{max}$. To this end, do the following.

\begin{itemize}
	\item Set the number of Monte Carlo repetitions $R=100$ and $p^{max}=8$.
	\item Initialize output matrices $aic$ and $sic$ each of dimension $R \times 5$. 
	\item For $r=1,...,R$ do the following:
	\begin{itemize}
		\item Simulate $10100$ observations for the DGP given above and discard the first 100 observations as burn-in phase. Save the remaining 10000 observations in a matrix $Y$.
		\item Compute the lag criteria for 5 different sample sizes $T=\{80, 160, 240, 500, 10000\}$, i.e. use the last $T$ observations of your simulated data matrix $Y$ for computations.
		\item Save the chosen lag order in the corresponding output object at position $[r,j]$ where $j=1,...,5$ indicates the corresponding sample size.
	\end{itemize}
	\item Look at the frequency tables of your output objects for the different subsamples. Hint: \texttt{tabulate(aic(:,1))} displays a frequency table for the AIC criterion with sample size equal to 80.
\end{itemize}
Given your results, do you agree with the following (general) findings?
\begin{enumerate}
	\item AIC is not consistent for the true lag order, whereas SIC is consistent.
	\item AIC never (asymptotically) selects a log order that is lower than the true lag order.
	\item In finite samples, we usually have $\hat{p}_{SIC} \leq \hat{p}_{AIC}$.
	\item In finite samples, AIC has a tendency to overestimate the lag order, SIC has a tendency to underestimate the lag order; hence, one should rely on AIC in finite samples.
\end{enumerate}

\newpage
\section{Posterior distribution of sign-identified structural IRFs}
Consider data for $y_t = (\Delta gnp_t,\Delta p_t,i_t)'$, where $gnp_t$ denotes the log of U.S. real GNP, $p_t$ the consumer price index in logs, and $i_t$ the federal funds rate. The sample period consists of 1970Q1 to 2011Q1.  Data is given in the Excel sheet \enquote{MonPolData} in the file \texttt{data.xlsx}.

\begin{enumerate}
	\item Estimate the parameters of a VAR(4) model with constant by using Bayesian methods, i.e. a Gibbs sampling method. To this end:
	\begin{itemize}
		\item Assume a Minnesota prior for the VAR coefficients, where the prior mean should reflect the view that the VAR follows a random walk. Set the hyperparameters for the prior covariance matrix $V_0$ such that the tightness parameter on lags of own and of other variables are both equal to 0.5, and the tightness parameter on the constant term is equal to 100. 
		\item Assume an inverse Wishart prior for the covariance matrix with degrees of freedoms equal to the number of variables and the identity matrix as prior scale matrix.
		\item Initialize the first draw of the covariance matrix with OLS values.
		\item Draw in total 30100 times from the conditional posteriors, where you discard draws of the coefficient matrix that are unstable.
		\item Save the last $n_{G}=100$, draws $(A^{r},\Sigma_u^{r})$ $(r=1,...,n_G)$ for inference on the structural model.
	\end{itemize}
	\item Estimate the posterior of the structural impulse response function by considering the following sign restrictions pattern on the impact matrix
		$$ B_0^{-1}=\begin{bmatrix}
		> & > & <\\
		> & < & <\\
		> & *  & >
		\end{bmatrix}$$
		 where $*$ denotes unrestricted values. That is, for each draw $(A^{r},\Sigma_u^{r})$ from the posterior of the reduced-form VAR Parameters:
		\begin{itemize}
		\item Compute the lower-triangular Cholesky decomposition $P^{r} = chol(\Sigma_u^{r})$.
		\item Compute a random draw of the rotation matrix $Q$.
		\item For each combination $(A^{r},P^{r},Q)$ compute the set of implied structural impulse responses $\Theta^{r}(h)$ with horizons $h=0,...,30$.
		\item If $\Theta^{r}(0)$ satisfies the sign restrictions on impact, store the value of $\Theta^{r}(h)$. Otherwise discard it.
		\item Repeat these steps until you have $n_Q=200$ accepted draws for each $(A^{r},\Sigma_u^{r})$.
		\end{itemize}
		Note that in the end you should have $n_Q\cdot n_{G}=20000$ accepted draws from the posterior of structural impulse response functions.
		\item Display the vector of point-wise posterior medians of the structural impulse responses, i.e. the median response function. 
		\item Interpret your results for one structural shock (of your choice) on (i) the level of real gnp, (ii) the consumer price index and (iii) on the federal funds rate.
		\item Name two shortcomings of using the median response function as a measure of central tendency in sign-identified SVARs.
\end{enumerate} 
\newpage
\section{How Well Does the IS-LM Model Fit Postwar US Data?}
Consider a quarterly model for $y_t = (\Delta gnp_t, \Delta i_t, i_t-\Delta p_t, \Delta m_t - \Delta p_t)'$, where $gnp_t$ denotes the log of GNP, $i_t$ the yield on three-month Treasury Bills, $m_t$ the growth in M1 and $p_t$ the inflation rate in the CPI. There are four shocks in the system: an aggregate supply (AS), a money supply (MS), a money demand (MD) and an aggregate demand (IS) shock. Ignoring the lagged dependent variables for expository purposes, the unrestricted structural VAR model can be written as  
\begin{align*}
\Delta gnp_t &= -b_{12}\Delta i_t -b_{13}(i_t-\Delta p_t) -b_{14}(\Delta m_t-\Delta p_t) + \varepsilon_t^{AS}\\
\Delta i_t &= -b_{21}\Delta gnp_t -b_{23}(i_t-\Delta p_t) -b_{24}(\Delta m_t-\Delta p_t) + \varepsilon_t^{MS}\\
i_t - \Delta p_t &= -b_{31}\Delta gnp_t -b_{32}\Delta i_t -b_{34}(\Delta m_t-\Delta p_t) + \varepsilon_t^{MD}\\
\Delta m_t - \Delta p_t &= -b_{41}\Delta gnp_t -b_{42}\Delta i_t - b_{43} (i_t-\Delta p_t) + \varepsilon_t^{IS}
\end{align*}
where $b_{ij}$ denotes the $ij$th element of $B_0^{-1}$. Consider the following identification restrictions:
\begin{itemize}
	\item Money supply shocks do not have contemporaneous effects on output growth, i.e. $$\frac{\partial \Delta gnp_t}{\partial \varepsilon_t^{MS}}=0$$
	\item Money demand shocks do not have contemporaneous effects on output growth, i.e. $$\frac{\partial \Delta gnp_t}{\partial \varepsilon_t^{MD}}=0$$
	\item Monetary authority does not react contemporaneously to changes in the price level, i.e. $$\frac{\partial \Delta i_t}{\partial \Delta p_t}=0$$
	\item Money supply shocks, money demand shocks and aggregate demand shocks do not have long-run effect on the log of real GNP, i.e. on the cumulated impulse response:
	$$\frac{\partial gnp_t}{\partial \varepsilon_t^{MS}}=0,\qquad \frac{\partial gnp_t}{\partial \varepsilon_t^{MD}}=0,\qquad \frac{\partial gnp_t}{\partial \varepsilon_t^{IS}}=0$$
	\item The structural shocks are assumed to have unit variance.
\end{itemize}
Solve the following exercises:
\begin{enumerate}
	\item Derive the implied exclusion restrictions on the matrices $B_0^{-1}$ and $\Theta(1)$.
	\item Consider the data given in the excel file \texttt{gali1992.xlsx}. Estimate a VAR(4) model with constant term.
	\item Estimate the structural impact matrix using a nonlinear equation solver, i.e. the objective is to find the unknown elements of $B_0^{-1}$ such that
	$$\begin{bmatrix}
	vech(B_0^{-1}B_0^{-1'}-\hat{\Sigma}_u)\\
	\text{short-run restrictions on }B_0^{-1}\\
	\text{long-run restrictions on }\Theta(1)\\
	\end{bmatrix}$$
	is minimized where the normalization $\Sigma_\varepsilon=I_3$ is imposed. Furthermore, normalize the shocks such that the diagonal elements of $B_0^{-1}$ are positive.
	\item Use the implied estimate of $B_0^{-1}$ to plot the structural impulse response functions for (i) real GNP, (ii) the yield on Treasury Bills, (iii) the real interest rate and (iv) real money growth.
\end{enumerate}

\newpage

\section{Inference In SVARs Identified By Exclusion Restrictions}

Consider an exactly-identified structural VAR model subject to short- and/or long-run restrictions, where the structural impulse response of variable $i$ to shock $j$ at horizon $h$ are simply denoted as $\theta \equiv \Theta_{ij,h}$. As an exact closed-form solution for the asymptotic standard errors of $\theta$ are only available under restrictive assumptions, we will rely on a numerical approximation.
\begin{enumerate}
	\item Reconsider an exercise (of your choice) from the lecture on SVAR models identified with exclusion restrictions and estimate the structural impulse response function.
	\item Write a Matlab function \texttt{bootstd(VAR,opt)} with the structure \texttt{VAR} from the reduced-form estimation step and corresponding options \texttt{opt} as inputs. The function computes $\widehat{std}(\hat{\theta}^\ast)$ via a bootstrap approximation, that is:
\begin{itemize}
	\item Set bootstrap repetitions $B$ equal to 1000 (or higher) and initialize a $K \times K \times H \times B$ array \texttt{THETAstar}, where the first dimension corresponds to variable $i=1,..,K$, the second dimension to shock $j=1,...,K$, the third dimension to the horizon of the IRFs $h=0,..,H$ and the fourth dimension to the bootstrap repetition $b=1.,...,B$.
	\item For $b=1,...,B$ do the following (you may also try \texttt{parfor} instead of \texttt{for} in order to make use of Matlab's parallel computing toolbox):
	\begin{itemize}
		\item Compute a bootstrap GDP $y_t^{b}$ using the function \texttt{BootstrapGDP(VAR)} which implements a standard residual-based bootstrap approach using sampling with replacement techniques on the residuals. Furthermore, the initial values are randomly drawn in blocks.
		\item Estimate the reduced-form and structural impulse response function on this artificial dataset with the same methodology, settings and identification restrictions as in the estimation on the original dataset.
		\item Store the structural IRFs in \texttt{THETAstar} at position \texttt{(:,:,:,b)}.

	\end{itemize}	
	\item Compute the standard deviation of the bootstrap structural IRFs using \texttt{std(THETAstart,0,4)} and use this as the output of your function \texttt{bootstd(VAR,opt)}.
\end{itemize}
\item Plot approximate 95\% confidence intervals for the structural impulse response functions according to the delta method:
\begin{align*}
\hat{\theta} \pm z_{\gamma/2} \widehat{std}(\hat{\theta}^\ast) \label{eq:deltairf}
\end{align*}
where $z_{\gamma/2}$ is the $\gamma/2$ quantile of the standard normal distribution.
\end{enumerate}
\end{document}
